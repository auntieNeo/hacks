\section*{\TeX{} / \LaTeX{} Hacks}
\subsection*{Including Sub Files}
This isn't much of a hack, but I learned it while making this document, and being meta is always fun, so here it is. To include sub files within a \TeX{} file, simply use the {\tt \textbackslash input} command, like this:
\begin{verbatim}
\input{blah.tex}
\input{foo/bar.tex}
\end{verbatim}
This seems to work something like C include files, but I could be mistaken.

\subsection*{Add Space After Symbol}
Since oftentimes one needs to place a symbol in a document without any space after it (e.g. when writing something like {\tt \textbackslash LaTeX} in a technical document about \LaTeX), most symbol macros don't seem to output a space when placed next to a word. For example, when we want to use \LaTeX{} in a sentence:
\begin{verbatim}
This is \LaTeX in a sentence.  -- This isn't what we want.
                               -- The output will look like
                               -- "LaTeXin a sentence."
\end{verbatim}
with most symbols this behavior is surprising. The solution is to use curly braces {\tt \{\}} like this:
\begin{verbatim}
This is \Latex{} in a sentence.  -- This is what we want.
\end{verbatim}
Some exceptions to this rule are the special symbols {\tt \textbackslash \&} and {\tt \textbackslash \$}, which behave as one would have initially expected.  % maybe this last point could be made a bit clearer... oh well
